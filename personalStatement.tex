

% \documentclass{nsfgrfp}
% \documentclass[timesfont]{nsfgrfp}
\documentclass[timesfont,runinheadings]{nsfgrfp}
% \documentclass[timesfont,runinheadings,crowdlines]{nsfgrfp}
% \documentclass[timesfont,runinheadings,crowdlinesmax]{nsfgrfp}
%
\usepackage{paralist}			% for compact lists
\usepackage[vskip=0.25ex]{quoting}	% for compact quotes

\begin{document}

% \section*{Research Statement.}

I used to be an arctic oceanographer in a previous life. I went on twelve expeditions, amounting to 218 days at sea, during which we saw the token polar bear, the occasional seal, but mostly the vast expanse of vanishing ice so poignantly portrayed by documentaries. We worked around the clock, methodologically piercing the dark blue surface, acquiring data that would later trickle down to contribute to our research on the ocean's role in our changing climate. My years as an oceanographer were instrumental in uncovering my affinity for data driven research, leading to my current position as a PhD candidate at the School of Computing (University of Utah) and my pursuit of a career as an academic in the field of information visualization. 

\section*{Intellectual Merit}
My first exposure to research dates back to my second year as an undergraduate at the University of S\~ao Paulo when I was awarded a research grant from FAPESP (S\~ao Paulo Research Foundation) to lead a study that correlated primary production in the Atlantic Ocean with satellite altimeter measurements of sea surface height. The project entailed extensive manipulation and analysis of large datasets, during which I found myself captivated by the potential to extract relevant information from these elaborate and often multidimensional datasets. I thought of data as a three-dimensional, oddly shaped, object. One that, when viewed at different angles, would reveal completely unique, often unexpected, insights and patterns. As an undergraduate I participated in several ocean science conferences and symposiums (see experience section), at which I presented posters and gave talks on my budding research. Upon graduation, I bypassed the job search predicament and merged seamlessly into a master’s degree at the School of Marine Science and Technology (SMAST) where I was awarded a grant to investigate the influence of the North Atlantic Oscillation (NAO) on temperature and salinity fields in the Gulf of Maine. In support of this research I explored computer science topics such as objective analysis methods and different visual frameworks for the temperature and salinity fields. One of the most valuable aspects of my work at SMAST was a greater understanding of computer science techniques as applied to the effective generation and visualization of geo-spatial fields.

After a brief stint in the industry sector writing ocean current prediction reports for oil and gas companies, I returned to academia working as a Research Associate (RA) at the Woods Hole Oceanographic Institution (WHOI). This position was the consummate fit for me as I was able to conduct research at a first class oceanographic institution while also exploring new techniques in information visualization, a burgeoning passion of mine. As part of my work I was directly responsible for the acquisition, processing, and analysis of physical measurements on oceanographic expeditions. This set the stage for my development of improved methodologies and interactive tools that would streamline these tasks, all ubiquitous to sea-going expeditions. Among these tools are CruisePlanner and OceanDepths, a pair of user interfaces that enable users to visualize, select, and process oceanographic data in real time. Cruise Planner is geared towards constructing a cruise sampling pattern while OceanDepths allows the user to view and interact with measurements of the ocean floor.  The resulting work was published as a paper at the IEEE VIS conference, in the industry practitioner’s track (Nobre, 2015). Both tools, first developed in 2013, are still used today by researchers both at WHOI and at the Geophysical Institute at the University of Bergen, Norway. These interfaces, built over the span of several scientific cruises, were the result of continuous feedback and refinements to the interaction mechanisms and desired functionality. Developing these and other tools at WHOI reinforced the importance of constant iteration and feedback when creating user interfaces, an essential skill for collaboration with domain experts.   

Another key aspect of my work at WHOI was the strong element of teamwork. Among my collaborations were projects on the Deep Western Boundary Current and the Gulf Stream (LineW), abyssal upwelling in the western North Atlantic (Dynamite), upwelling events  in the Chukchi sea  (AON), circulation patterns  in the Denmark strait  (K\"ogur project), and, most recently, water mass variability and upwelling in Barrow Canyon (DBO). While my contributions to most of these projects were within the scope of data processing and figure generation, the Barrow Canyon research was arguable on of my largest scientific contributions during my time at WHOI.  The study on water mass variability in the Pacific Arctic was made possible with funding I received from NOAA’s Arctic Research Program. I presented the results at a talk in Ocean Sciences 2014 and am submitting a paper on our latest analysis to the Journal of Oceanography.  

Almost two years into my time at WHOI, I found myself pushing against the limits of my technical expertise. There was a growing discrepancy between the tools I ideated to streamline data acquisition and analysis, and my technical ability to implement them. I embraced the challenge and in the fall of 2012 embarked on what would later become the conceptual basis for my research in information visualization. The Harvard Extension School’s Master Program in Software Engineering offered a selection of course that were remarkably germane to my work at WHOI with data manipulation, visualization, and creating user-data interfaces. During the following three years I fulfilled the program requirements on evenings and weekends while maintaining a 4.0 GPA and my full-time position as an RA at WHOI. For my thesis, advised by Dr.Alexander Lex, I developed OceanPaths, a browser based tool which provides powerful interaction and exploration methods for spatial, multivariate oceanographic datasets. The project involved extensive research into existing multivariate data visualization in the realm of earth sciences as well as close collaboration with other oceanographers to guide the visual encodings and interaction mechanisms. Despite the existence of previous studies that addressed multi-variable datasets in earth sciences, OceanPaths presented a unique research contribution with a novel solution for reducing the dimensionality of spatial oceanographic data by condensing geographic coordinates into a network of meaningful paths. The work led to a publication (Nobre and Lex, 2015) and a talk at the 2015 EuroVis Conference in Cagliari, Italy where the paper was a runner up for the best short paper award. Being the lead on a project involving both oceanography and computer science experts gave me valuable experience on working effectively in an interdisciplinary team, an undisputed quality in a field known for cross-domain collaborations.

\section*{Eligibility}
This section is devoted to justifying the graduate degree received at Harvard as necessary to maintain my employment as a Research Associate at WHOI. As demonstrated by the two publications relating to OceanPaths, and later CruisePlanner and OceanDepths, the skill set I  developed at Harvard was instrumental in various aspects of my work at WHOI, including the development of tools that facilitated the research being done by various PIs, both  inside  and  outside  of the institution.  Moreoever, a large part of working with scientists is the constant call for innovation, improvement and progress. While my first year was fraught with opportunities to employ the skillset I had at the time, I quickly found myself looking for ways of improving the way we processed, distributed, and visualized these large datasets. As an example, two of of the Harvard classes I took were specifically geared towards  dynamic web  applications and were  essential in my role as  web  developer/data manager for the DBO and K\"ogur projects, the latter involving an international team of investigators. My consistently top ratings in the yearly evaluations at WHOI serve as further evidence that my education at Harvard was a key component to performing my job at an optimum level.

\section*{Broader Impacts} 
George Bernard Shaw once said \textit{The single biggest problem in communication is the illusion that it has taken place}. As anyone in science can attest, proper communication, whether it be through teaching, writing, or speaking, has long been a challenging facet of research. Growing up in a bilingual household I learned this at an early age, along with the cultural awareness afforded a child who moved overseas nine times during her formative years. This experience, however, was foundational for my later years as a teacher. Starting at the age of 18, I organized and taught English as a Second Language (ESL) classes for students, staff and faculty at the University of S\~ao Paulo. Teaching English for over five years was instrumental in cementing my ability to communicate complex concepts to different audiences, a skill set which I now rely on heavily. I continued teaching during my time at WHOI, where I mentored students in data processing, visualization, and sofware development to aid in their research. Among my mentees were WHOI/MIT joint program students, guest students, and interns that participated on research expeditions. My teaching trajectory culminated this fall with a Teaching Mentor position for Dr. Lex's Visualization class at the University of Utah. As a mentor I have found great satisfaction giving lectures, developing homeworks, and providing support for students as they navigate the learning curve associated with new and often challenging topics. My teaching experience over the last 10 years has provided a solid foundation in cogently conveying information to audiences, an essential skill in effectively communicating research. 

Aside from teaching, I am also passionate about programs targeted towards increasing interest and engaging young girls in the topics of Science, Technology, Engineering and Math (STEM).  One such program that I was involved with was GIRLS, which brings together top women in their respective fields of Science and Engineering with young girls from the local community. I was an active participant in the 2013 edition, including a talk on my career at WHOI delving on my path from a girl in a small town in Brazil to my involvement with academia today. My involvement with the GIRLS program taught me two things: the importance of reducing the unwarranted stigma of 'girls and science' by sparking interest in the STEM fields early on, and the value in providing positive role models of women who have trailed that path successfully.  

\section*{Future Goals } 
I chose to attend the University of Utah because of the research developed at the Visualization Data Lab, specifically the work led by Dr.Alexander Lex, my advisor. Dr.Lex's research interest align closely with mine and focuses on innovative, interactive, visual data analysis methods that leverage both, the power of computation and the unique abilities of humans to interpret large and complex datasets. My PhD research, described further in the Graduate Research Statement, investigates a novel approach in visualizing and analyzing large and multidimensional datasets. Our proposal focuses on tree and graph networks, a type of data often found in social networks,  biological pathways, and genealogical trees. As such, there is a wide range of applications for the proposed work, including a better understanding of genetic factors in certain disesases. After completing the PhD program, I plan on continuing my academic efforts, focusing on novel ways of visualizing and extracting actionable information from complex datasets, all the while leveraging human contextual knowledge.

Looking back at my career path I see that, albeit unorthodox, my trajectory has equipped me with a unique set of qualifications in pursuing an academic career. My experiences as a graduate student and as a research associate, have shown me that I find the process of research rewarding. That it is in the intricacies of understanding problem-driven research that I find myself challenged and ultimately rewarded. My path here has been multifaceted and as such has afforded me a unique vantage point – a perspective bolstered by a wealth of cross-disciplinary efforts, all leading to a more holistic understanding of the field. Moreover, my accumulated experiences have been instrumental in establishing the set of skills that I have come to rely on the most as a researcher: effective communication, collaboration amongst peers, and intellectual curiosity.

As I progress in my academic career I am confident that my contributions to the field will be a result not only of my keen interest in the topic but the diverse and solid skill set that I acquired along the way. I am also hopeful that as a woman from a developing country, I can lead by example, encouraging girls and women to embrace careers in the STEM fields.   

\end{document}

