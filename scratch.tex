The strength of our proposed approach lies in exploiting the delicate balance between human 
 
 Understanding and extracting actionable information from the growing amount of large and complex datasets is an ongoing and largely unsolved challenge in the world of information visualization research. Our approach , a solution to which could significantly advance our comprehension of several social, geographic, and biological phenomena. An effective solution must be faithful not only to the topology of the data but must also be able to efficiently represent the set of attributes associated to the data. Our proposed solution explores linearizing graphs and trees in a way that preserves the topology of the network while providing an associated table based view of the data attributes. One of the main advantages of this approach is the ability to compare and analyze both topology and attributes at the same time. This solution also allows for aggregation of nodes based on their degree of interest, thus making the layout significantly more scalable than existing solutions. Our proposed solution would benefit applications ranging from the analysis of social networks to a better understanding of biological networks which would significantly improve our comprehension of genetic diseases.
 
 

Alex's research statement for reference


My research is at the interface of humans and data. Data and analytical methods have become ubiquitous, yet human reasoning and contextual knowledge are critical to discovery and decision making. I develop innovative, interactive, visual data analysis methods that leverage both, the power of computation and the unique abilities of humans to interpret large and complex datasets. Humans and computers have complementary strengths: humans are creative, have an unsurpassed visual pattern discovery system, and have often rich knowledge about the data they interpret. Algorithms and computers are fast, easily available, and precise. Yet, developing seamless and effective interfaces between humans and large and complex datasets is challenging and in many cases an open research question which I address.


At the risk of reaching too far back into the past, I credit my childhood, specifically my parents, for an early start in the awareness of developing these skills. Growing up in a bilingual household, my siblings and I were exposed .  – 
Points: cultural diversity, understanding of “the world”


Paragraph on growing up in two countries, speaking two languages, adjusting to different cultures.  At the risk of sounding clichê, I credit my parents for an early start in developing the aforementioned skills. 

Paragraph on college. Teaching English. ESL/TOEFL/IELTS training programs. 
Learning to communicate. To teach. To break down what is easy in one’s mind to someone else. Start of research career. FAPESP grant. 

Paragraph on Master’s degree in 2010. 
More research experience. 

2-3 paragraphs on Woods Hole. Talk about Harvard Degree and make sure to use wording from the RFP to justify a graduate degree within the last 2 years. 

Cemented team work, critical thinking, and increased interest in communication. Link the desire to effectively communicate knowledge to the interest in using visualization to explore large and complex data sets. Give OceanPaths and oceandepths papers as examples. Talk about leveraging human knowledge to extract information from large datasets. 


proactive and not reactive. skill sets that I developed while at WHOI -> harvard extension school -> data vis
- ability to continuosly develop a skill set at WHOI , not only established needs but needs that I perceived
-data integration @ sea - link to intellectual curiousity 

segway from collaboration to curiousity, at sea, data integration needs, leading to...

necessary to maintain my 5 star ranking within the department. 
sentence on how Harvard shifted my focus. 

team work
after FAPES 
collaborating within an academic setting throughout college and my first masters. 

team work in an industry setting

at sea teamwork

data integration 

But when you're standing on the deck of a research vessel in the middle of the Arctic Ocean, there is no crescendo music that cues the feeling of impending doom on our fragile planet. No close-up of a polar bear and her cubs, or the calving iceberg, all narrated by Morgan Freeman's foreboding voice as he warns us of the point of no return.

I tell you this not in an attempt to soften your hearts to a vanishing ice land, but to 

In Fall 2016 I began my PhD at the University of Utah School of Computing, concentrating in graphics and visualization, as well as a research assistantship at the University of Utah’s Scientific Computing and Imaging (SCI) Institute under Professor Alexander Lex. Professor Lex’s research focuses on innovative, interactive, visual data analysis methods that leverage both, the power of computation and the unique abilities of humans to interpret large and complex datasets. His research aligns very well with the nature of the work I have conducted thus far and am interested in developing during my PhD.

\section*{Collaborative Work/teamwork}
\begin{itemize}
  \item collaboration in academic setting (masters)
  \item collaboration in industry setting (horizon)
  \item collaboration in research institution (WHOI)
  \item Segway to next section:  collaboration  at sea highlights a collaboration within a larger framework of data collection in general. Sparked intellectual curiosity of how to improve data exploration.  
\end{itemize}

\section*{Intellectual Curiosity}
\begin{itemize}
\item working at WHOI with this newfound intellectual curiosity.
\item in doing so, noticed that I did not have the sufficient knowledge to significantly improve our ability to explore and visualize oceanographic datasets. (i.e, bridge the gap that i had perceived)
\item in the pursuit of pushing knowledge to the next level,i was awarded 5* at every evaluation cycle , and in order to continue in that trajectory, i would need to enhance my skill set. 
\item As a result, my boss and I agreed that it would be essential to the institution that I pursue a master''s degree at Harvard. 
\item That experience was a challenge, coordinating school and work, but maintained 4.0 GPA.
\item published two papers on interactive data tools applied to oceanography 
\item In turn, the whole Harvard experience further enhanced my curiosity to delve into the data visualization and exploration leading me to start
a PhD in the area
\end{itemize}